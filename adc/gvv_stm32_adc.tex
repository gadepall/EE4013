\documentclass[journal,12pt,twocolumn]{IEEEtran}
%
\usepackage{setspace}
\usepackage{gensymb}
\usepackage{xcolor}
\usepackage{caption}
%\usepackage{subcaption}
%\doublespacing
\singlespacing

%\usepackage{graphicx}
%\usepackage{amssymb}
%\usepackage{relsize}
\usepackage[cmex10]{amsmath}
\usepackage{mathtools}
%\usepackage{amsthm}
%\interdisplaylinepenalty=2500
%\savesymbol{iint}
%\usepackage{txfonts}
%\restoresymbol{TXF}{iint}
%\usepackage{wasysym}
\usepackage{amsthm}
\usepackage{mathrsfs}
\usepackage{txfonts}
\usepackage{stfloats}
\usepackage{cite}
\usepackage{cases}
\usepackage{subfig}
%\usepackage{xtab}
\usepackage{longtable}
\usepackage{multirow}
%\usepackage{algorithm}
%\usepackage{algpseudocode}
\usepackage{enumitem}
\usepackage{mathtools}
\usepackage{iithtlc}
%\usepackage[framemethod=tikz]{mdframed}
\usepackage{hyperref}
\usepackage{listings}
    \usepackage[latin1]{inputenc}                                 %%
    \usepackage{color}                                            %%
    \usepackage{array}                                            %%
    \usepackage{longtable}                                        %%
    \usepackage{calc}                                             %%
    \usepackage{multirow}                                         %%
    \usepackage{hhline}                                           %%
    \usepackage{ifthen}                                           %%
  %optionally (for landscape tables embedded in another document): %%
    \usepackage{lscape}     

\usepackage{tikz}
\usepackage{url}
\def\UrlBreaks{\do\/\do-}

%\usepackage{stmaryrd}


%\usepackage{wasysym}
%\newcounter{MYtempeqncnt}
\DeclareMathOperator*{\Res}{Res}
%\renewcommand{\baselinestretch}{2}
\renewcommand\thesection{\arabic{section}}
\renewcommand\thesubsection{\thesection.\arabic{subsection}}
\renewcommand\thesubsubsection{\thesubsection.\arabic{subsubsection}}

\renewcommand\thesectiondis{\arabic{section}}
\renewcommand\thesubsectiondis{\thesectiondis.\arabic{subsection}}
\renewcommand\thesubsubsectiondis{\thesubsectiondis.\arabic{subsubsection}}

% correct bad hyphenation here
\hyphenation{op-tical net-works semi-conduc-tor}

%\lstset{
%language=C,
%frame=single, 
%breaklines=true
%}

%\lstset{
	%%basicstyle=\small\ttfamily\bfseries,
	%%numberstyle=\small\ttfamily,
	%language=Octave,
	%backgroundcolor=\color{white},
	%%frame=single,
	%%keywordstyle=\bfseries,
	%%breaklines=true,
	%%showstringspaces=false,
	%%xleftmargin=-10mm,
	%%aboveskip=-1mm,
	%%belowskip=0mm
%}

%\surroundwithmdframed[width=\columnwidth]{lstlisting}
\def\inputGnumericTable{}                                 %%
\lstset{
language=C,
frame=single, 
breaklines=true
}
 

\begin{document}
%

\theoremstyle{definition}
\newtheorem{theorem}{Theorem}[section]
\newtheorem{problem}{Problem}
\newtheorem{proposition}{Proposition}[section]
\newtheorem{lemma}{Lemma}[section]
\newtheorem{corollary}[theorem]{Corollary}
\newtheorem{example}{Example}[section]
\newtheorem{definition}{Definition}[section]
%\newtheorem{algorithm}{Algorithm}[section]
%\newtheorem{cor}{Corollary}
\newcommand{\BEQA}{\begin{eqnarray}}
\newcommand{\EEQA}{\end{eqnarray}}
\newcommand{\define}{\stackrel{\triangle}{=}}

\bibliographystyle{IEEEtran}
%\bibliographystyle{ieeetr}

\providecommand{\nCr}[2]{\,^{#1}C_{#2}} % nCr
\providecommand{\nPr}[2]{\,^{#1}P_{#2}} % nPr
\providecommand{\mbf}{\mathbf}
\providecommand{\pr}[1]{\ensuremath{\Pr\left(#1\right)}}
\providecommand{\qfunc}[1]{\ensuremath{Q\left(#1\right)}}
\providecommand{\sbrak}[1]{\ensuremath{{}\left[#1\right]}}
\providecommand{\lsbrak}[1]{\ensuremath{{}\left[#1\right.}}
\providecommand{\rsbrak}[1]{\ensuremath{{}\left.#1\right]}}
\providecommand{\brak}[1]{\ensuremath{\left(#1\right)}}
\providecommand{\lbrak}[1]{\ensuremath{\left(#1\right.}}
\providecommand{\rbrak}[1]{\ensuremath{\left.#1\right)}}
\providecommand{\cbrak}[1]{\ensuremath{\left\{#1\right\}}}
\providecommand{\lcbrak}[1]{\ensuremath{\left\{#1\right.}}
\providecommand{\rcbrak}[1]{\ensuremath{\left.#1\right\}}}
\theoremstyle{remark}
\newtheorem{rem}{Remark}
\newcommand{\sgn}{\mathop{\mathrm{sgn}}}
\providecommand{\abs}[1]{\left\vert#1\right\vert}
\providecommand{\res}[1]{\Res\displaylimits_{#1}} 
\providecommand{\norm}[1]{\lVert#1\rVert}
\providecommand{\mtx}[1]{\mathbf{#1}}
\providecommand{\mean}[1]{E\left[ #1 \right]}
\providecommand{\fourier}{\overset{\mathcal{F}}{ \rightleftharpoons}}
%\providecommand{\hilbert}{\overset{\mathcal{H}}{ \rightleftharpoons}}
\providecommand{\system}{\overset{\mathcal{H}}{ \longleftrightarrow}}
	%\newcommand{\solution}[2]{\textbf{Solution:}{#1}}
\newcommand{\solution}{\noindent \textbf{Solution: }}
\providecommand{\dec}[2]{\ensuremath{\overset{#1}{\underset{#2}{\gtrless}}}}
%\numberwithin{equation}{subsection}
\numberwithin{equation}{problem}
%\numberwithin{problem}{subsection}
%\numberwithin{definition}{subsection}
\makeatletter
\@addtoreset{figure}{problem}
\makeatother

\let\StandardTheFigure\thefigure
%\renewcommand{\thefigure}{\theproblem.\arabic{figure}}
\renewcommand{\thefigure}{\theproblem}


%\numberwithin{figure}{subsection}

%\numberwithin{equation}{subsection}
%\numberwithin{equation}{section}
%%\numberwithin{equation}{problem}
%%\numberwithin{problem}{subsection}
\numberwithin{problem}{section}
%%\numberwithin{definition}{subsection}
%\makeatletter
%\@addtoreset{figure}{problem}
%\makeatother
\makeatletter
\@addtoreset{table}{problem}
\makeatother

\let\StandardTheFigure\thefigure
\let\StandardTheTable\thetable
%%\renewcommand{\thefigure}{\theproblem.\arabic{figure}}
%\renewcommand{\thefigure}{\theproblem}
\renewcommand{\thetable}{\theproblem}
%%\numberwithin{figure}{section}

%%\numberwithin{figure}{subsection}



\def\putbox#1#2#3{\makebox[0in][l]{\makebox[#1][l]{}\raisebox{\baselineskip}[0in][0in]{\raisebox{#2}[0in][0in]{#3}}}}
     \def\rightbox#1{\makebox[0in][r]{#1}}
     \def\centbox#1{\makebox[0in]{#1}}
     \def\topbox#1{\raisebox{-\baselineskip}[0in][0in]{#1}}
     \def\midbox#1{\raisebox{-0.5\baselineskip}[0in][0in]{#1}}

\vspace{3cm}

\title{ 
	\logo{
STM32 ADC
		}
}



% paper title
% can use linebreaks \\ within to get better formatting as desired
%\title{Matrix Analysis through Octave}
%
%
% author names and IEEE memberships
% note positions of commas and nonbreaking spaces ( ~ ) LaTeX will not break
% a structure at a ~ so this keeps an author's name from being broken across
% two lines.
% use \thanks{} to gain access to the first footnote area
% a separate \thanks must be used for each paragraph as LaTeX2e's \thanks
% was not built to handle multiple paragraphs
%

\author{G V V Sharma$^{*}$% <-this % stops a space
\thanks{*The author is with the Department
of Electrical Engineering, Indian Institute of Technology, Hyderabad
502285 India e-mail:  gadepall@iith.ac.in. All content in this manual is released under GNU GPL.  Free and open source.}% <-this % stops a space
%\thanks{J. Doe and J. Doe are with Anonymous University.}% <-this % stops a space
%\thanks{Manuscript received April 19, 2005; revised January 11, 2007.}}
}
% note the % following the last \IEEEmembership and also \thanks - 
% these prevent an unwanted space from occurring between the last author name
% and the end of the author line. i.e., if you had this:
% 
% \author{....lastname \thanks{...} \thanks{...} }
%                     ^------------^------------^----Do not want these spaces!
%
% a space would be appended to the last name and could cause every name on that
% line to be shifted left slightly. This is one of those "LaTeX things". For
% instance, "\textbf{A} \textbf{B}" will typeset as "A B" not "AB". To get
% "AB" then you have to do: "\textbf{A}\textbf{B}"
% \thanks is no different in this regard, so shield the last } of each \thanks
% that ends a line with a % and do not let a space in before the next \thanks.
% Spaces after \IEEEmembership other than the last one are OK (and needed) as
% you are supposed to have spaces between the names. For what it is worth,
% this is a minor point as most people would not even notice if the said evil
% space somehow managed to creep in.



% The paper headers
%\markboth{Journal of \LaTeX\ Class Files,~Vol.~6, No.~1, January~2007}%
%{Shell \MakeLowercase{\textit{et al.}}: Bare Demo of IEEEtran.cls for Journals}
% The only time the second header will appear is for the odd numbered pages
% after the title page when using the twoside option.
% 
% *** Note that you probably will NOT want to include the author's ***
% *** name in the headers of peer review papers.                   ***
% You can use \ifCLASSOPTIONpeerreview for conditional compilation here if
% you desire.




% If you want to put a publisher's ID mark on the page you can do it like
% this:
%\IEEEpubid{0000--0000/00\$00.00~\copyright~2007 IEEE}
% Remember, if you use this you must call \IEEEpubidadjcol in the second
% column for its text to clear the IEEEpubid mark.



% make the title area
\maketitle

\tableofcontents


% creates the second title. It will be ignored for other modes.
\IEEEpeerreviewmaketitle

\bigskip

\begin{abstract}
This manual shows how to interface the $16 \times 2$ HD44780-controlled LCD using STM32F103C8T6.
\end{abstract}
%\newpage
\section{Components}
\begin{table}[!h]
%\footnotesize
\input{./figs/components.tex}
\caption{Components}
\label{table:components}
\end{table}


%\begin{problem}
%List all the available timers in the STM32F103C8T6 blue pill.
%\end{problem}
%\solution  See Table \ref{table:stm32_timers}
%\begin{table}[!h]
%\footnotesize
%\input{./figs/stm32_timers.tex}
%\caption{STM32F103C8T6 Timer Types.}
%\label{table:stm32_timers}
%\end{table}
\section{Internal Temperature Sensor}
\begin{problem}
Make connections as shown in Table \ref{table:pins}.
\end{problem}
\begin{table}[!h]
\footnotesize
\input{./figs/pins.tex}
\caption{Pin Connections}
\label{table:pins}
\end{table}
\begin{problem}
Execute the following program
\begin{lstlisting}
https://github.com/gadepall/STM32F103C8T6/blob/master/examples/adc/internal_temp.c
\end{lstlisting}
What do you observe?
\end{problem}
\solution You should observe a number between 1750-1760. This is the output of the internal temperature
sensor, captured in $ADC1->DR$.
\begin{problem}
Find an expression for $V_{SENSE}$
\end{problem}
\solution
\begin{equation}
V_{SENSE} = 3.3 \times \frac{ADC1->DR}{4095}
\end{equation}
%
\begin{problem}
Obtain the formula for finding the temperature of the STM32 and list the values of the various
parameters.
\end{problem}
\solution The desired formula is
\begin{equation}
T = (V_{25}-V_{SENSE})/AvgSlope + 25
\end{equation}
where the typical values of the above parameters are 
\input{./figs/sensor.tex}
%
\begin{problem}
What is the default ADC frequency?
\end{problem}
%
\solution The ADC operates at 14 MHz by default and is indpendent of the processor frequency (8 MHz in this case). It can, however be synchronized with
the processor clock for some real time applications.
\begin{problem}
Explain the significance of the following instruction
\begin{lstlisting}
ADC1->SMPR1 |= ADC_SMPR1_SMP16;
\end{lstlisting}
\end{problem}
\solution Through this command, $ADC1->SMPR1$ = 0x001C0000 where the SMPR1 register is shown in Fig. \ref{fig:smpr1}.  Note that
this makes SMP16 = 111 which means that channel 16 sample time = 239.5 cycles.  Channel 16 is reserved for the internal temperature 
sensor and is connected to ADC1.
%
\begin{figure}
\centering
\includegraphics[width=\columnwidth]{./figs/smpr1.eps}
\caption{SMPR1}
\label{fig:smpr1}
\end{figure}
%
\begin{problem}
What is the sampling time?
\end{problem}
\solution Since the sample time is 239.5 cycles and the ADC frequency is 14 MHz, 
\begin{equation}
T_{s} = 239.5 \times \frac{1}{14} \mu s = 17.1 \mu s 
\end{equation}
\begin{problem}
Explain the following instruction.
\begin{lstlisting}
ADC1->SQR3 |= ADC_SQR3_SQ1_4;
\end{lstlisting}
\end{problem}
\solution ADC\_SQR3\_SQ1\_4 = 0x00000010.  This implies that SQ1=0b10000 in
the ADC regular sequence register 3 ($ADC1->SQR3$)shown in Fig. \ref{fig:sqr3}. Since SQ1=16,
this means that the ADC input in channel 16 will be the first in the queue
for conversion. 
%
\begin{figure}
\centering
\includegraphics[width=\columnwidth]{./figs/sqr3.eps}
\caption{SQR3}
\label{fig:sqr3}
\end{figure}
%
The ADC is capable of converting analog 16 inputs one after the other. 
The inputs are called {\em channels} and the sequence number corresponding 
to the channel
is decided
according to the 5 bit entry in SQ. 
\begin{problem}
Configure SQR3 so that the 9th channel for ADC1 is 2nd in sequence.
\end{problem}
\solution This implies that SQ2=1001.  Thus,
\begin{equation}
ADC1->SQR3 = 0x000000120
\end{equation}
\section{Measuring an Unkown Resistance}
\begin{problem}
List the various pin numbers corresponding to the different channels 
of the ADC.
\end{problem}
\solution See Fig. \ref{table:pin_channel}
\begin{table}[!h]
\footnotesize
\input{./figs/pin_channel.tex}
\caption{ADC Analog Input Pins}
\label{table:pin_channel}
\end{table}
\begin{problem}
Use the 9th channel of  ADC1 in SQ2 to measure 3.3V. 
\end{problem}
\solution connect PB1 to 3.3 V of the STM 32 and execute the following
code.
\section{Project}
\begin{problem}
Measure an unkown resistance using the STM32 and display the result 
on the LCD.
\end{problem}
\begin{problem}
Display the output of the internal temperature sensor as well
the unknown resistance on the LCD.
\end{problem}
\end{document}


